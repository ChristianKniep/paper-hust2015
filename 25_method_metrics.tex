\subsubsection{Metrics Engine}
The first pillar of the open framework is the metric engine ecosystem Graphite, which captures performance values over time. 
First released in 20xx it provides a modular system to consume, process, store and access metric information in the form 'key value time-stamp'.

The metrics are persisted in round-robin databases called whisper by a set of daemons called 'carbon'.

\begin{lstlisting}[language=bash,
    caption={Sending and receiving metrics using carbon in combination with a whisper file back-end.},
    label={lst:carbon_wsp}]
$ for x in {1..5};do
> echo "test.metric ${x} $(date +%s)" |\
    nc -w 1 127.0.0.1 2003
> sleep 5
> done
$ export CDIR=/var/lib/carbon/whisper
$ whisper-fetch $CDIR/test/metric.wsp |\
    grep -v None
1438513460	1.000000
1438513465	2.000000
1438513470	3.000000
1438513475	4.000000
1438513480	5.000000
$
\end{lstlisting}

The graphite framework provides a rich ecosystem of visualizations, plugins, metric sources and alternative daemon implementation to fit almost
all use-cases around metrics gathering, processing, storing and serving metrics. Due to its open-source development and defined API missing pieces
can be added if needed.

This enables each part of the infrastructure layer to submit it's metrics. Even application developers can just start sending metrics down the pipeline and
let them pop up at the access layer.

- Metric Key Scheme

If the group of metric generators grows to a certain complexity the metrics keys becoming a problem, since naming schemes vary depending on the context of
the producer. An engineer overseeing the network in a data-center might organize his metrics hierarchical ($datacenter01.rack01.compute01.eth0.send\_bytes$), while from the
resource scheduler perspective a grouping per job ($jobid01.compute01.send\_bytes$) is a scheme which might fit the use-case best.

To overcome this issue a scheme should be chosen which describes the metric in a deterministic and unique way. This ID is then logically structured according to the use-case.
The concept behind Metrics 2.0 e.g. adds metadata (stored in an elasticsearch index) to each unique identifiable key, which allows for context based searches.

