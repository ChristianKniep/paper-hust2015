\section{Results}
By combining the presented technology stacks, QNIBMonitoring provides deep insights into the complete cluster stack.

\subsection{Domain Modeling}
By using a schema-less graph database approach, each domain can be modeled independently. Common entities are connected at any time after
they are modeled.

\subsubsection{Resource Scheduler}
\gls{slurm} provides information about resource allocations: Which jobs was started where, by whom at what time. How long did it last? What was the outcome? and alike.
The following \gls{slurm} entities are modeled as nodes within the graph.
\begin{description}
    \item[host] The SLURM node, which represents a compute node within the cluster, identified by the host-name.
    \item[partition] A logical group of nodes which represent the target of a job submission.
    \item[job] A demand for resources submitted by a user. Including metadata (user, submission/start/end time, job script, etc) and information about the state of the job.
        If resources are allocated, the job metadata is updated accordingly.
\end{description}
Relationships are added to connect certain nodes. $PART_OF$ declares host memberships to partitions. $JOBCLIENT$ connects a host to a certain running job.

\subsubsection{InfiniBand Interconnect}
An InfiniBand representation comprises of a set of physical and logical representations.
\begin{figure}[!ht]
    \includegraphics[width=.4\textwidth]{images/png/infiniband_graph.png}
    \caption{\label{fig:ib_graph}Modeling of an InfiniBand network.}
\end{figure}

\begin{description}
    \item[HCA] Network adapter plugged into a host/
    \item[SW] InfiniBand switch
    \item[PORT] Each $HCA$ and $SW$ are composed of a set of ports which provide connectivity.
    \item[CABLE] Physical connection between different ports.
    \item[CONNECTION] Logical entity which connects two $PORTS$ and a $CABLE$ for a given period in time.
\end{description}

Relationships combine two $PORTS$ and a $CABLE$ alongside with meta-information about the time when it was created, if the connection is still valid and a like.
Furthermore the routing information is added by providing a $ROUTE_TO$ relationship between a port and a host.

\subsection{Holistic Inventory}
The presented two models alone provide combined insight into the cluster. By connecting the HCA (within the InfiniBand model) to a host (within the SLURM model) the inventory allows inter-layer queries.

\subsubsection{Job Placements}
By showing the layout of a job within the \gls{ib} topology, a imbalanced job is easily detectable. Figure X shows an example in which the hop-count within a job differs.
%\begin{figure}[!ht]
%    \includegraphics[width=.4\textwidth]{images/png/hop_missmatch.png}
%    \caption{\label{fig:ib_graph}Modeling of an InfiniBand network.}
%\end{figure}
An slightly more complex case would include different port-characteristics like speed, width.
By caching information of the metrics system within the ports a even more sophisticated query might include error or performance counters.

\subsubsection{InfiniBand Debugging}
A common debug case within InfiniBand network is to figure out if errors associated with a connection is caused by one of the ports or the cable itself.
The approach to fix this error is to carefully unplug one connector of a given cable, connect it to a different switch-port and check the error counters of the just established connection.
If the error sticks with the cable the oposite side of the cable has to be changed as well, to truly pin down the problem.

This process is extremly error prone; an inventory to deal with an InfiniBand fabric barely exists. Therefore each step is manual labor with the risk of miscommunication between the on-site engineer changing the
cables and the engineer matching the changes to performance counter changes.

By using an holistic inventory the changes are tracked automatically and since they include the timestamps the correlation with the metrics systems is trivial.

\subsection{Open Log and Metric Framework}
Determining inter-layer error triggers is an extremly hard task. If e.g. a specific job triggers an error within the interconnects one has to keep track of the job placement, the routes within the interconnect and
the time to correlate metrics with this events.

%\begin{figure}[!ht]
%    \includegraphics[width=.4\textwidth]{images/png/hop_missmatch.png}
%    \caption{\label{fig:ib_graph}Modeling of an InfiniBand network.}
%\end{figure}
Fig Y shows the performance metrics annotated with job events, which provides an easy to spot correlation between the two domains.