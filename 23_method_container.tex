The current wave of OS-level virtualization leverage Kernel Namespaces \footnote{\Mundus~\url{http://man7.org/linux/man-pages/man7/namespaces.7.html}} to isolate different groups of processes. Unlike hypervisor virtualization, which emulates
the hardware of a physical computer, OS-level virtualization spawns processes with certain flags, attaching the process (and its children's processes)
to a set of namespaces. The Linux Kernel provides Kernel Namespaces to encapsulate processes, mount points, network stacks,
inter-process communication and domain information. Furthermore a container can be granted only a subset of kernel capabilities.

A common way to restrict the use of resources (CPU, memory, I/O) available by a running container is the use of \gls{cgrp} (\glsdesc{cgrp}).

\subsubsection{Linux Containers}
An early contender to OS-level virtualization FreeBSD `jails', Solaris `Zones' and \gls{lc}. They never gained much momentum, since they had operational limitations are were a bit cumbersome to use.

\glspl{lc} are a perfect fit for a microservice architecture. They do not spin up a complete operating system but only the process
necessary for the service and therefore minimize the overhead. Furthermore \glspl{lc} are started and stopped within a
fraction of a second, which provides a proactive elasticity needed to be agile.

\subsubsection{Docker}
The \gls{lc} implementation that took off within the last couple of months is Docker. It leads the rise of \glspl{lc} as a commodity technology, accessible
for a broad range of users. It provides an easy and intuitive way to build, distribute and run container images.
By developing Docker in the open and providing an API, Docker gained a lot of momentum and surrounded Docker with a rich ecosystem.

\subsubsection{Containers versus Virtual Machines}
One could argue that virtual machines are best suited to serve monolithic applications,
which assume an uninterrupted operation. If the underlying hardware is failing the virtual machine gets migrated to a different host. Ideally just
with a slight hiccup, while the network address is transferred. The application itself is designed to life a long and protected life.

In the microservice architecture dealing with failure is a crucial part of the design itself. A system should not have single points of failure and
should self-heal if a service fails. How many (and which) concurrent errors a system can withstand depends on the effort put into it and might differ
in certain stages of the development, but the ideal goal is a self-healing system.


