The motivation behind this work dates back to the urge to run a complete HPC cluster stack in a visualized fashion on a single laptop in order to be able to gain a better understanding of how a HPC software stack interacts.
%This first attempt was presented during the \gls{osdc} (\glsdesc{osdc}) .

All services were put into a distinced machines, which only served an environment needed to run a given service.
Without calling it microservices, the idea behind it was quite similar with the approach microservices are associated with.
One of the most quoted definitions of the term microservices is the following\footnote{\Mundus~\url{http://www.slideshare.net/adriancockcroft/microxchg-microservices}}:
\epigraph{microservices is defined as a loosely coupled service oriented architecture with bounded context}{\textit{Adrian Cockcroft}}
In practice the emerging pattern is that small teams get full ownership of a particular service (bounded context).
How the service is implemented is of no interest to the clients (most likely microservices themselves) of this service.
The interaction is strictly defined by a contract. A common protocol to provide this interaction is a RESTful API based on HTTP and JSON.

A litmus test of such an architecture is an update of services. `Loosely coupled` should lead to a solution in which an update of one service
should not enforce an update to services using the API. If this test is passed successfully the system provides
independent update cycles for each service, without breaking the current system as a whole.

