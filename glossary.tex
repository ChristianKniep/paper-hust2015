\makeglossaries

\newacronym{hpc}{HPC}{'High Performance Computing'}
\newacronym{ib}{IB}{'InfiniBand'}
\newacronym{ui}{UI}{'User Interface'}
\newacronym{mpi}{MPI}{'Message Passing Interface'}
\newacronym{lxc}{LXC}{'LinuX Containers'}
\newacronym{nfs}{NFS}{'Network File System'}
\newacronym{hpcg}{HPCG}{'High Performance Conjugate Gradient'}
\newacronym{rhel}{RHEL}{'RedHat Enterprise Linux'}
\newacronym{cae}{CAE}{'Computer Aided Engineering'}
\newacronym{cfd}{CFD}{'Computational Fluid Dynamics'}
\newacronym{hpcac}{HPCAC}{'HPC Advisory Council'}
\newacronym{sysop}{SysOp}{'System Operations'}
\newacronym{openfoam}{OpenFOAM}{'Open Source Field Operation and Manipulation'}


%% Glossary
\newglossaryentry{kibana}{name=Kibana, description={Client-side javascript visualisation framework\footnote{\Mundus~\url{https://www.elastic.co/products/kibana}}}}
\newglossaryentry{lc}{name=LinuX Container, plural=LinuX Containers, description={\footnote{\Mundus~\url{https://en.wikipedia.org/wiki/LXC}}}}
\newglossaryentry{skydns}{name=\texttt{SkyDNS}, description={DNS interface to \gls{etcd}\footnote{\Mundus~\url{https://github.com/skynetservices/skydns1}}}}
\newglossaryentry{consul}{name=\texttt{Consul}, description={distributed cluster management\footnote{\Mundus~\url{https://www.consul.io/}}}}
\newglossaryentry{etcd}{name=\texttt{etcd}, description={distributed key/value store\footnote{\Mundus~\url{https://github.com/coreos/etcd/}}}}
\newglossaryentry{confd}{name=\texttt{confd}, description={Configuration management tool using templates\footnote{\Mundus~\url{http://www.confd.io/}}}}
\newglossaryentry{slurm}{name=SLURM, description={Simple Linux Utility for Resource Management\footnote{\Mundus~\url{http://slurm.schedmd.com/}}}}
\newglossaryentry{elasticsearch}{name=Elasticsearch, description={Open-source text-index engine based on Lucene\footnote{\Mundus~\url{https://www.elastic.co/products/elasticsearch}}}}
\newglossaryentry{splunk}{name=Splunk, description={a commercial platform for Operational Intelligence\footnote{\Mundus~\url{http://www.splunk.com/}}}}
\newglossaryentry{graylog}{name=Graylog, description={Log management platform\footnote{\Mundus~\url{https://www.graylog.org/}}}}
\newglossaryentry{cgrp}{name=CGroups, description={Linux Control Grroups\footnote{\Mundus~\url{https://www.kernel.org/doc/Documentation/cgroups/}}}}
\newglossaryentry{graphite}{name=Graphite, description={Metrics engine\footnote{\Mundus~\url{http://graphite.readthedocs.org/en/latest/}}}}
\newglossaryentry{logstash}{name=Logstash, description={powerful, modular event pipeline\footnote{\Mundus~\url{http://logstash.net/}}}}
\newglossaryentry{neo4j}{name=Neo4J, description={a graph database based on Lucene\footnote{\Mundus~\url{http://neo4j.com/}}}}
\newglossaryentry{osdc}{name=OSDC2014, description={Open Source Datacenter Conference 2014\footnote{\Mundus~\url{https://www.netways.de/?id=3132\#c27028}}}}
\newglossaryentry{ec}{name=EC, description={error code}}
\newglossaryentry{fqdn}{name=FQDN, description={Fully Qualified Domain Name}}
\newglossaryentry{nagios}{name=NAGIOS, description={Health check and alerting system widely used within the IT\footnote{\Mundus~\url{https://www.nagios.org/}}}}
\newglossaryentry{tsdb}{name=OpenTSDB, description={\footnote{\Mundus~\url{http://opentsdb.net/}}}}
\newglossaryentry{influxdb}{name=InfluxDB, description={\footnote{\Mundus~\url{https://influxdb.com/}}}}
\newglossaryentry{openmp}{name=OpenMP, description={Open Multi-Processing, is an API that provides a shard memory among multiple servers.}}
\newglossaryentry{omp}{name=Open MPI, description={oMPI}}
\newglossaryentry{cos6}{name=CentOS 6.5, description={Community Enterprise Operating System in version 6.5, derived from the sources of Red Hat Enterprise Linux 6 (RHEL6).}}
\newglossaryentry{cos7}{name=CentOS 7.0, description={Community Enterprise Operating System in version 7.0, derived from the sources of Red Hat Enterprise Linux 7 (RHEL7).}}
\newglossaryentry{u12}{name=Ubuntu 12.04, description={Ubuntu 12.04.}}
\newglossaryentry{qdr}{name=QDR, description={Quad data rate (or quad pumping) is a communication signaling
        technique wherein data are transmitted at four points in the clock cycle:
        on the rising and falling edges, and at two intermediate points between them.}}

\newglossaryentry{rpm}{name=RPM, description={RPM Package Manager, a package management system created by RedHat. }}

\newglossaryentry{sriov}{name=SR-IOV, description={oMPI}}
