\begin{abstract}
The IT infrastructure (no matter the size) is composed of multiple layers, supplied by different commercial vendors as well
as more and more open-source project. This increasing complexity needs to be addressed by defining a common sense framework
to consume, process and explore the data produced by any part of the system; no matter if it is metrics data, event information and inventory data.

The innovation cycle is becoming faster and faster and thus the need to integrate new services need to keep pace.

In this paper a system is introduced, which serves as an open framework to consume, process, store and make use of
bit and pieces of information originating from each layer of the infrastructure stack.
The framework utilizes Docker as a foundation of a microservice architecture and makes use of the best-of-breed open-source technologies
such as \gls{graphite} (\glsdesc{graphite}), \gls{logstash} (\glsdesc{logstash}) and \gls{neo4j} (\glsdesc{neo4j}).
This approach provides an inter-layer methodology which allows connections between subsystems like network, software inventory and resource scheduler to
assist in complex analysis.
\end{abstract}
