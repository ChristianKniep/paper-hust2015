\begin{abstract}
IT infrastructure (no matter the size) is composed of multiple layers, supplied by different commercial vendors as well
as more and more open-source projects. This increasing complexity needs to be addressed by defining a common sense framework
to consume, process and explore the data produced by any part of the system; no matter if it is metrics data, event information or inventory data.

The pace of innovation is becoming faster and faster and thus the integration of new services has to keep up to speed. Long integration cycles
are going to be a handicap in a competitive changing world.

In this paper a system is introduced, which serves as an open framework to consume, process, store and make use of
bits and pieces of information originating from each infrastructure layer.
The framework uses Docker as a foundation of a microservice architecture and integrates the best-of-breed open-source technologies
such as \gls{graphite}, \gls{logstash} and \gls{neo4j}.
This approach provides an inter-layer methodology which allows connections between subsystems like network,
software inventory and resource scheduler to assist in complex analysis of problems and performance, for example by correlating Infiniband
performance metrics with job-related information obtained from SLURM.
\end{abstract}
